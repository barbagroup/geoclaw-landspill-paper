%! TEX root = main.tex
\section{Models and numerical methods}

The full shallow-water equations (SWE), which are derived from depth-averaged Navier-Stokes equations and belong to conservation laws, are used to model the oil overland flow problems:
\begin{equation}\label{eq:swe}
    \pd{}{t}\vec{q} + \pd{}{x}\vec{f}(\vec{q}) + \pd{}{y}\vec{g}(\vec{q}) = \vec{\psi}(\vec{q}, x, y, t)
\end{equation}
where $\vec{q} = \begin{bmatrix} h \\ hu \\ hv \end{bmatrix}$,
$\vec{f}(\vec{q}) = \begin{bmatrix} hu \\ hu^2 + \frac{1}{2}gh^2 \\ huv \end{bmatrix}$,
$\vec{g}(\vec{q}) = \begin{bmatrix} hv \\ huv \\ hv^2 + \frac{1}{2}gh^2 \end{bmatrix}$, and
$\vec{\psi}(\vec{q}, x, y, t) = \begin{bmatrix} R-I \\ -ghB_x - F_x \\ -ghB_y - F_y \end{bmatrix}$.
Table \ref{table:notation} describes the meaning of each symbol.

\begin{table}
    \caption{Notation}
    \begin{tabular*}{\tblwidth}[t]{p{0.175\columnwidth}p{0.75\columnwidth}}
        \toprule
        Symbol & Description \\
        \midrule
        $x$, $y$, $t$ & Spatial and temporal coordinates. \\
        $g$ & Gravitational acceleration. \\
        $h$ & Fluid depth. \\
        $u$, $v$ & Depth-averaged velocity in $x$- and $y$-directions. \\
        $\vec{q}$ & Vector of conservative quantities. \\
        $\vec{f}(\vec{q})$, $\vec{g}(\vec{q})$ & Vector of fluxes in $x$- and $y$-directions. They are functions of vector $\vec{q}$. \\
        $\vec{\psi}(\vec{q}, x, y, t)$ & Source terms. \\
        $R$ & Depth (mass) increasing rate. \\
        $I$ & Depth (mass) removal rate. \\
        $B_x$, $B_y$ & Gradient components of topography in $x$- and $y$-directions. \\
        $F_x$, $F_y$ & Bottom friction components in $x$- and $y$-directions. \\
        \bottomrule
    \end{tabular*}
    \label{table:notation}
\end{table}

SWE has been used to model overland flow problems such as rainfall-runoff problems, dam-break problems, tsunami simulations, and river flow simulations.
The major difference between oil overland flow and the aforementioned applications lies in the rheological properties of the working fluids.
While the SWE in equation \ref{eq:swe} is derived from incompressible Navier-Stokes equations for Newtonian fluids, hydrocarbon products are not always Newtonian fluids.
Even for some Newtonian hydrocarbon fluids, their viscosities are sensitive to the environment variables, such and temperature and pressure.
Nevertheless, at the current stage of this project, we still model the oil flow with the SWE in equation \ref{eq:swe}.
According to Rønningsen (\cite{Ronningsen2012}) and Bryan~et~al. (\cite{bryan_viscosity_2002}), crude oils are normally Newtonian, though the viscosity highly depends on temperature. 

In SWE, lateral friction is ignored.
And all viscous effects are lumped into the bottom friction model, which calculates $F_x$ and $F_y$.
