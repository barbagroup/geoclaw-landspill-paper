%! TEX root = main.tex
\subsection{Full shallow-water equations}

We model the oil overland flow with the full shallow-water equations (SWE), which are derived from depth-averaged Navier-Stokes equations (\cite{vreugdenhil_numerical_1994}):
\begin{equation}\label{eq:swe}
    \pd{}{t}\vec{q} + \pd{}{x}\vec{f}(\vec{q}) + \pd{}{y}\vec{g}(\vec{q}) = \vec{\psi}(\vec{q}, x, y, t)
\end{equation}
where $\vec{q} = \begin{bmatrix} h \\ hu \\ hv \end{bmatrix}$,
$\vec{f}(\vec{q}) = \begin{bmatrix} hu \\ hu^2 + \frac{1}{2}gh^2 \\ huv \end{bmatrix}$,
$\vec{g}(\vec{q}) = \begin{bmatrix} hv \\ huv \\ hv^2 + \frac{1}{2}gh^2 \end{bmatrix}$, and
$\vec{\psi}(\vec{q}, x, y, t) = \begin{bmatrix} R-I \\ -ghB_x - F_x \\ -ghB_y - F_y \end{bmatrix}$.
Table \ref{table:notation} describes the meaning of each symbol.

\begin{table}
    \caption{Notation}
    \begin{tabular*}{\linewidth}[t]{p{0.11\linewidth}p{0.06\linewidth}p{0.7\linewidth}}
        \toprule
        Symbol & Unit & Description \\
        \midrule
        $x$, $y$ & $m$ & Spatial coordinates. \\
        $t$ & $s$ & Temporal coordinates. \\
        $g$ & $m/s^2$ & Gravitational acceleration. \\
        $h$ & $m$ & Fluid depth. \\
        $u$, $v$ & $m/s$ & Depth-averaged velocity in $x$- and $y$-directions. \\
        $R$, $I$ & $m/s$ & Mass source and sink. \\
        $B_x$, $B_y$ & none & Gradient components of topography in $x$- and $y$-directions. \\
        $F_x$, $F_y$ & $m^2/s^2$ & Bottom friction components in $x$- and $y$-directions. \\
        \bottomrule
    \end{tabular*}
    \label{table:notation}
\end{table}

Shallow-water equations are derived under the assumption of incompressible flow and Newtonian fluids. 
Though not all hydrocarbon products are Newtonian fluids, crude oils normally are. (\cite{Ronningsen2012, bryan_viscosity_2002}) 
In this work, our main focus is the crude oil pipelines. 

Equation \ref{eq:swe} is also called the dynamic shallow-water equations because it considers the effects of local acceleration, inertia, pressure, gravity, and viscosity.
While many variants and simplified shallow-water equations exist, we argue that the full shallow-water equations are necessary to describe the overland flow in pipeline incidents.
The gravity effect takes place everywhere because of topographical changes.
The pressure effect also applies to everywhere as long as the water surface elevation is not horizontal.
The inertia is at least significant at the vicinity of a pipeline rupture point and at the beginning of a rupture incident: the high pressure difference between inside and outside a pipe generates high flow speed at a rupture point.
Lastly, the viscosity effect is lumped into the bottom friction and is at least significant to flow front and after pipeline valves are shut down, causing flow speed to be slower.

George and Berger (\cite{George2011, Berger2011}) provide detailed descriptions of the numerical methods, especially the adaptive-mesh-refinement (AMR) algorithms, used in \geoclaw.
Built on top of \geoclaw, our work focuses on adding models for mass sources ($R$), mass sinks ($I$), bottom friction ($F_x$ and $F_y$), and temperature-dependent viscosity.
Our work also adds features that helps practical use cases in real-world analysis, such as standard CF-compliant NetCDF I/O, Esri ArcGIS integration, and cloud computing integration with Microsoft Azure.
